\subsection{Феномен тесного мира}

Для того, чтобы понять, каким должен быть граф для 
оптимальной работы нашего алгоритма, я предлагаю обратиться
к структурам, которые возникают само собой в приро\-де, 
к графам, которые формирет общество.

\subsubsection{Виденье Stanley Milgram}
Stanley Milgram - американский социальный психолог и педагог.
в своей статье The Small World Problem ~\cite{TheSmallWorldProblem}
Милгрэм рассуждает на тему гипотезы 6 рукопожатий, которая гласит, что 
любые два человека на планете могут быть связаны через 6-х
знакомых. Он обосновывал правдивость этой гипотезы тем, что наше
общество состоит из кластеров. То есть, каждый из нас имеет примерно 500
близких знакомых, каждый из которых имеет ещё 500 знакомых(другие кластеры)
и так далее. Если посмотреть на полученную геометрическую прогрессию, мы
получаем число $ 500^6 = 1,56 * 10^{16} $. Даже с учётом того факта, что
кластеры могут пересекаться, настолько большой результат выглядит 
очень убедительно

Для подтверждения своей модели, Милгрэм провёл эксперимент
Он выдал 300 писем людям из разных городов и попросил доставить 
их одному человеку из Бостона (США). 
По результатам исследование, даже не смотря на то, что до человека-цели
дошли далеко не все письма (Милгрэм оправдал это тем, что люди имели
недостаточно информации о цели, поэтому иногда принимали не оптимальные
решения), те, которым это удалось, прошли в среднем через цепь из
5-6 человек. 

Однако, при попытке смоделировать данное поведение на компьютере, даже
на небольшой выборке значений, это не сработает. Если мы построим
граф, который будет состоять из кластеров вершин (вершины, соеденины
ребром только в случае, если они находятся рядом), его диаметр будет
всё равно очень большим. То есть, чтобы добраться от одной вершины
к другой, понадобиться много посредников

\subsubsection{Случайные графы Erdős–Rényi}
Попробуем подойти к проблеме с другой стороны и предположить, что
графы на самом деле полностью случайны. Для простоты будем считать,
что вероятность проведения каждого ребра одинаково и выбирается так,
чтобы произведение кол-ва вершин n на неё было фиксировано и равно
средней степени вершин. Так мы описали граф, изучением которого занимался
Венгерский математик Erdős–Rényi. 

Первый вопрос, который может возникнуть: а будет ли такой граф вообще
связен? Ведь если нет, то говорить о "кол-ве рукопожатий" \ вовсе не 
будет иметь никакого смысла. Оказывается, что да. Erdős доказал, что

\begin{equation}
    c \ge 3, \ n \ge 300, \ p = \frac{c\ln{n}}{n} \Rightarrow \mathbb{P}
    (\text{G - связен}) \xrightarrow{a.s.} 1
\end{equation}

Что буквально говорит о том, что при увеличении выборки вероятность
связности графа только растёт

Второй же вопрос тоже назревает сам собой: Чем данный подход может
превосходить предыдущий? Идея заключается в том, что так как вероятность 
рёбер никак не зависит от расстояния между вершинами, будет 
появляться значительное кол-во связей на средние и далёкие расстояния, что
будет способствовать быстрому перемещению по графу.

Однако такой подход тоже не идеален, проблема возникла в другом. Теперь
мы легко можем добраться от одной части графа к другой, но столкнёмся
с тем, что из-за отсутствия большого кол-ва локальных связей поиск тех 
или иных вершин будет часто упираться в локальные минимумы из которых не 
будет выхода

\subsubsection{Тесные графы}

Часто истина кроится по середине, поэтому в современное обоснование
проблемы 6 рукопожатий говорит о том, что знакомства в обществе можно
представить в виде графа со следующими свойствами:

\begin{itemize}
    \item У каждой вершины есть большое кол-во близких связей 
(решают проблему локаль\-ных минимумов)
    \item У каждой вершины есть ограниченый набор длинных вершин
(решают проблемы с навигацией по графу) 
\end{itemize}

Формализуем данные понятия, чтобы в дальнейшем было удобно 
сравнивать эти ха\-рактеристики численно:















