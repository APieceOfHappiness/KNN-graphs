\subsection{Описание абстракций}

Для начала, я бы хотел описать, какие классы были реализованы, 
каким образом и для чего они преднозначены.

\subsubsection{Класс Graph}

Данный граф является основным. Он реализует все самые необходимые инструменты для
взаиможействия с графами, такие как: добавление вершины в граф, добавление ребра, очистка графа
и т.д. 

Данная абстракция была реализована на основе хэш таблицы (std::unordered\_map). 
Такой контейнер был выбран не случайно, так как он осуществляет очень быстрый доступ 
к произвольным данным (Добавление, поиск, удаление за O(1)).

Этот класс является шаблонным и может хранить в себе объекты любой природы. Главное, чтобы
они удовлетворяли двум требованиям:

\begin{itemize}
    \item Наличие перегрузки оператора равенства 
    \item Наличие функции хэширования (для std::unordered\_map)
\end{itemize}
Объявление класса: 

\begin{verbatim}
    template<typename TObject, typename HashFunc>
    class Graph {
        std::unordered_map<TObject, std::vector<TObject>, HashFunc> graph;
        std::vector<TObject> nodes;
        size_t size;
    public:
        Graph() = default;
        Graph(const Graph& g) = delete;
        Graph(Graph&& g) = delete;

        void add_node(const TObject& p);
        void add_edge(const TObject& p1, const TObject& p2);
        void delete_edge(const TObject& p1, const TObject& p2);
        void clear();
        std::size_t get_size() const;
        bool consists_node(const TObject& p) const;
        bool consists_edge(const TObject& p1, const TObject& p2) const;
        const std::vector<TObject>& get_neighbours(const TObject& p) const;
        const TObject& get_random_node() const;
        friend std::ostream& operator<< <>(std::ostream& out, const Graph& graph);
        ~Graph() = default;
    };
\end{verbatim}
Хочу отдельное внимание обратить на метод get\_random\_node. Его необходимость 
станет очевидна вдальнейшем, пока мне бы хотелось рассказать о дилеме, с которой я столкнулся
во время её написания. Для реализации этого метода, мы должны иметь возможность выбирать из хэш таблицы
любой ключ случайно, однако это возможно сделать только за линейное время, из-за отсутствия в 
unordered\_map обращения по индексу. Для решения этой проблемы мною было принято решение создать отдельный
вектор из объектов, так как в нём, используя случайные индексы, мы без проблем сможем выбрать 
абсолютно любой объект за O(1). 
Это привело к следующим последствиям:
\begin{itemize}
    \item Увеличение кол-во памяти необходимое для хранения структуры
    \item Ухудшение асимптотической скорости удаление выбранных элементов (до линейной)
\end{itemize}
Однако все эти недостатки никак негативно не повлияли конкретно в решении задач текущего проекта 


\subsubsection{Классы Point и Point3D}

Данные абстракции в моём проекте используются, как объекты, вершины графа. 
Они являются больше примером для наглядной демонстрации работы других классов.
Заголовки выглядят следующим образом:


\begin{verbatim}
    class Point {
        public:
            double x;
            double y;
            Point() = default;
            Point(const Point& p) = default;
            Point(Point&& p) = default;
            Point(double x, double y);
            friend std::ostream& operator<<(std::ostream& out, const Point& p);
            bool operator==(const Point& p) const;
            static double dist(const Point& p1, const Point& p2);
            class HashPoint {
            public:
                std::size_t operator()(const Point& p) const;
            };
            ~Point() = default;
        };
        class Point3D {
        public:
            double x;
            double y;
            double z;
            Point3D() = default;
            Point3D(const Point3D& p) = default;
            Point3D(Point3D&& p) = default;
            Point3D(double x, double y, double z);
            friend std::ostream& operator<<(std::ostream& out, const Point3D p);
            bool operator==(const Point3D p) const;
            static double dist(const Point3D p1, const Point3D p2);
            class HashPoint {
            public:
                std::size_t operator()(const Point3D& p) const;
            };
        };
\end{verbatim}

\subsubsection{Класс Nswg} 


Данный класс является абстрактным. Внутри он содержит всё необходимое для реализации
жадного поиска, однако на этом всё. Процесс формирования самого графа ложится на плечи
потомков (поэтому метод load\_nodes помечен, как виртуальный). Данное решение было принято, чтобы не копировать в каждый новый класс одинаковые 
методы жадного поиска.

\begin{verbatim}
    template<typename TObject, typename HashFunc>
    class Nswg {
    protected:
        Graph<TObject, HashFunc> graph;
        std::size_t sum_of_degrees = 0;

        class ClosestToCompare {
            const TObject& base;
        public:
            ClosestToCompare(const TObject& base) : base(base) {};
            bool operator()(const TObject& p1, const TObject& p2) const;
        };

        void multi_search(const TObject& target_node, 
                          std::set<TObject, ClosestToCompare>& res, 
                          std::size_t count) const;

    public:
        virtual void load_nodes(const std::vector<TObject>& objects) = 0;
        double get_mean_deg() const;
        double get_cc() const;

        const std::vector<TObject>& get_nodes() const;

        const TObject& greedy_search(const TObject& target_node, const TObject& start_node) const;
        std::pair<TObject, double> get_best_element(const TObject& target_node, std::size_t count) const;

        const std::vector<TObject>& get_neighbours(const TObject& p) const;
        const TObject& get_random_node() const;
        void clear();
        std::size_t get_size() const;
    };
\end{verbatim}

Также, хотелось бы ещё рассказать почему я использую внутренние дополнительный класс ClosestToCompare(). 
Как я уже описывал, в жадном поиске нам нужно искать k вершин наиболее близких с нашей.
Постоянные сортировки могли достаточно сильно ухудшить производительность модели, поэтому я принял решение, 
написать предикат, который будет сортировать вершины в set() сразу так, как нужно:
\begin{itemize}
    \item В конструктор класса мы передаём объект v
    \item Чем ближе вершина к v, тем первее она будет находится в set.
    \item Однако возникнет проблема, что 2 узла на одном расстоянии теперь будут считаться
    одинаковыми, а значит, в set добавится только один из них. Для решения этой проблемы, мне пришлось
    добавить второе условие, котороые сравнивает хеши вершин и сортирует на основе их значений:
\end{itemize}

\begin{verbatim}
    template<typename TObject, typename HashFunc>
    bool Nswg<TObject, HashFunc>::ClosestToCompare::operator()(const TObject& p1, const TObject& p2) const {
        double d1 = TObject::dist(p1, base);
        double d2 = TObject::dist(p2, base);
        if (d1 != d2) {
            return TObject::dist(p1, base) < TObject::dist(p2, base);
        } else {
            HashFunc hash;
            return hash(p1) < hash(p2);
        }
    }
\end{verbatim}

\subsubsection{NNGraph}

Является наследником класса Nswg и с помощью метода load\_nodes реализует построение графа.
Испрользует подход математиков из Нижнего Новгорода, упомянутых выше
Алгоритм построения уже был разобран в блок схеме выше.

\begin{verbatim}
    template<typename TObject, typename HashFunc>
    class NNGraph : public Nswg<TObject, HashFunc> {
        std::size_t queries_count = 5;
        std::size_t edges_count = 7;
    public:
        NNGraph(std::size_t queries_count = 5, std::size_t edges_count = 7);
        // NNGraph(const std::vector<TObject>& objects);

        NNGraph(const NNGraph& pg) = delete;
        NNGraph(NNGraph&& pg) = delete;

        void add_node(const TObject& obj);
        void load_nodes(const std::vector<TObject>& objects) override;

        friend std::ostream& operator<< <>(std::ostream& out, const NNGraph<TObject, HashFunc>& pgraph);

        ~NNGraph() = default;
    };
\end{verbatim}


\subsubsection{KleinbergGraph}

Является наследником класса Nswg и с помощью метода load\_nodes реализует построение графа.
Алгоритм построения уже был разобран в блок схеме выше.

\subsubsection{RandomGraph}

Данный класс был реализован на основе графов Erdos-Renye. Был создан для сравнительного анализа

\begin{verbatim}
    template<typename TObject, typename HashFunc>
    class RandomGraph : public Nswg<TObject, HashFunc> {
        double mean_neighbours;
    public:
        RandomGraph(double mean_neighbours);
        RandomGraph(const RandomGraph& rg) = delete;
        RandomGraph(RandomGraph&& rg) = delete;

        void load_nodes(const std::vector<TObject>& objects) override;

        ~RandomGraph() = default;
    };
\end{verbatim}

\subsubsection{Figure}

Данная абстракция была создана с целью упрощения визуализации графов и графиков.
Он работает по очень простому принципу: Под капотом она пишет скрипт на питоне, который
записывает в отдельный файл. После чего в отдельном процессе этот файл запускается и 
выводит данные в виде графиков. В Питоне я использовал библиотеку Plotly


\begin{verbatim}
    class Figure {
        std::ofstream script;
        std::string name;
        void write_vector_as_list(const std::string& name, const std::vector<double> &data);
    public:
        Figure() = delete; // TODO: why not
        Figure(const std::string& name);
        Figure(const Figure& f) = delete;
        Figure(Figure&& f) = delete;
        void add_graph(const Nswg<Point, Point::HashPoint>& g,
                       std::size_t marker_size,
                       std::size_t line_width,
                       const std::string& marker_color,
                       const std::string& line_color,
                       const std::string& name);
        void update_title(const std::string& title);
        void add_markers_and_lines(const std::vector<double>& x,
                                   const std::vector<double>& y,
                                   std::size_t marker_size,
                                   std::size_t line_width,
                                   const std::string& marker_color,
                                   const std::string& line_color,
                                   const std::string& name);
        void add_markers(const std::vector<double>& x,
                         const std::vector<double>& y,
                         std::size_t marker_size,
                         const std::string& marker_color,
                         const std::string& name);
        void add_line(const std::vector<double>& x,
                      const std::vector<double>& y,
                      std::size_t line_width,
                      const std::string& line_color,
                      const std::string& name);
        void update_xaxes(const std::string& x_title, 
                          const std::string& axis_type);
        void update_yaxes(const std::string& y_title, 
                          const std::string& axis_type);
        void show();
        ~Figure();
    };
\end{verbatim}




