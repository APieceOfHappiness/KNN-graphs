\subsection{Описание абстракций}

Для начала, я бы хотел описать, какие классы были реализованы, 
каким образом и для чего они преднозначены.

\subsubsection{Класс Graph}

Данный граф является основным. Он реализует все самые необходимые инструменты для
взаиможействия с графами, такие как: добавление вершины в граф, добавление ребра, очистка графа
и т.д. 

Данная абстракция была реализована на основе хэш таблицы (std::unordered\_map). 
Такой контейнер был выбран не случайно, так как он осуществляет очень быстрый доступ 
к произвольным данным (Добавление, поиск, удаление за O(1)).

Этот класс является шаблонным и может хранить в себе объекты любой природы. Главное, чтобы
они удовлетворяли двум требованиям:

\begin{itemize}
    \item Наличие перегрузки оператора равенства 
    \item Наличие функции хэширования (для std::unordered\_map)
\end{itemize}
Объявление класса: 

\begin{verbatim}
    template<typename TObject, typename HashFunc>
    class Graph {
        std::unordered_map<TObject, std::vector<TObject>, HashFunc> graph;
        std::vector<TObject> nodes;
        size_t size;
    public:
        Graph() = default;
        Graph(const Graph& g) = delete;
        Graph(Graph&& g) = delete;

        void add_node(const TObject& p);
        void add_edge(const TObject& p1, const TObject& p2);
        void delete_edge(const TObject& p1, const TObject& p2);
        void clear();
        std::size_t get_size() const;
        bool consists_node(const TObject& p) const;
        bool consists_edge(const TObject& p1, const TObject& p2) const;
        const std::vector<TObject>& get_neighbours(const TObject& p) const;
        const TObject& get_random_node() const;
        friend std::ostream& operator<< <>(std::ostream& out, const Graph& graph);
        ~Graph() = default;
    };
\end{verbatim}
Хочу отдельное внимание обратить на метод get\_random\_node. Его необходимость 
станет очевидна вдальнейшем, пока мне бы хотелось рассказать о дилеме, с которой я столкнулся
во время её написания. Для реализации этого метода, мы должны иметь возможность выбирать из хэш таблицы
любой ключ случайно, однако это возможно сделать только за линейное время, из-за отсутствия в 
unordered\_map обращения по индексу. Для решения этой проблемы мною было принято решение создать отдельный
вектор из объектов, так как в нём, используя случайные индексы, мы без проблем сможем выбрать 
абсолютно любой объект за O(1). 
Это привело к следующим последствиям:
\begin{itemize}
    \item Увеличение кол-во памяти необходимое для хранения структуры
    \item Ухудшение асимптотической скорости удаление выбранных элементов (до линейной)
\end{itemize}
Однако все эти недостатки никак негативно не повлияли конкретно в решении задач текущего проекта 


\subsubsection{Классы Point и Point3D}

Данные абстракции в моём проекте используются, как объекты, вершины графа. 
Они являются больше примером для наглядной демонстрации работы других классов.
