\newpage
\section{Введение}

\begin{justify}

    В последнее десятилетие наша жизнь стала тесно связана с 
    удобными приложениями и сервисами. Многие из них базируются 
    на рекомендательных системах, для предоставления целевого 
    товара на основе наших интересов. Некоторые используют 
    компьютерное зрение для решения огромного кол-ва задач 
    (от масок в социальных сетях, до автопилота в электроавтомобилях). 
    Поиск синонимов и системы автоматического дополнения текста(T9)
    также используются каждый день миллионами пользователей. Это лишь
    малая часть задач, которые можно решить используюя алгоритм
    поиска ближайшего соседа (или поиска K-ближайших соседей).

    Вот ещё некоторые задачи, о которых хотелось бы упомянуть:
    \begin{itemize}
        \item Поиск дубликатов (определить являются ли 2 текстовых документа
        одинаковыми)
        \item Задача кластеризации (Определить, как какой группе относится 
        выбранный объект)
        \item Поиск ближайших географических объектов (карты)
        \item Поиск схожих фрагментов в фильмах или музыке
    \end{itemize}

    Именно поэтому так важно искать новые подходы для улучшения 
    скорости данного алгоритма. Чтобы достичь поставленную цель, 
    необходимо разработать структуру данных, которая сможет наиболее
    эффективно осуществлять две операции добавления и поиска. Вариантов 
    подходящих структур - огромное множество. Например, некоторые
    могут быть построены на базе вектора, списка, дерева, графа (в
    виде сети). Некоторые поддерживают точные поиск, а некоторые только
    приближённый. Некоторые формируются по средствам детерминированных
    алгоритмов, а некоторые использоуют рандомизированный подход. 
    
    Моё исследование буддет в основном основываться на изучении
    графовых структур (в виде сетей), так как они более современные
    и эффективные (подробнее ниже). Перед собой я ставлю следующие 
    задачи:
    \begin{itemize}
        \item Изучить какие структуры для поиска ближайшего соседа существуют
        % \item сравнить их асимптотику построения этих структур и поиска 
        % внутри них, выявить преимущества и недостатки 
        \item Исследовать феномен тесного мира.
        \item Обосновать выбор именно графовых структур, а также подробнее 
        исследовать те из них, которые имеют свойствр тесного мира.
        % \item Выдвинуть несколько предположений о модификациях, которые
        % смогли бы улучшить имеющиеся методы.
        \item Разработать необходимые абстракции для работы с подобными
        структурами
        \item Провести сравнительный анализ
        % \item А также проверить эффективность предложенные ранее модификации
    \end{itemize}    
    
    
\end{justify}


