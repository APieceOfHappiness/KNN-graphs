\subsection{Подход Клайнберга}

    Следующий подход предложил Американский математик Джон Клайнберг. Перед началом разбора,
мне бы хотелось сделать некоторую поправку. Джон Клайнберг в своих статьях нигде не 
упоминает о коэффиценте класстеризации. Вместо этого он говорит только о наличии длинных
и коротких рёбер. Это немного расширяет класс "Тесных" графов, однако идейно 
ничего не меняет.

    Изучим заново граф, который изображён на Рис. \ref{lattice}. Он состоит только из
коротких связей, а значит, тесным графом не является. Есть разные способы решить эту проблему,
давайте рассмотрим предложение Джона Клайнберга, так как его метод формирования длинных
рёбер очень естественно вписывается в окружающий нас мир. Идея проста: чем человек дальше от меня, тем меньше
вероятность нашего знакомства. Формально его идею можно записать так, 
Графу даются 2 параметра p, k:
\begin{itemize}
    \item p - радиус. Все вершины, расстояние между которыми меньше или равно p, мы соединяем ребром и 
    и называем короткой связью
    \item k - кол-во длинных связей. После добавления коротких связей, останется только пройтись всевозможным 
    парам вершин и расставить рёбра в зависимости от следующего распределения:
\end{itemize}

\begin{equation} \label{edges_distribution}
    P(u \rightarrow v) = \frac{1}{d(u, v)^r}\frac{1}{C}, \text{ где } C = \sum_{u \neq z \in V}\frac{1}{d(u, z)^r}
\end{equation}


\begin{figure}[H]
    \centering
    \includegraphics[scale=0.3]{./pictures/random_graph.png}
    \caption{Блок схема построение графа по методу Джона Клайнберга} \label{Kleinberg_graph_block_scheme}
\end{figure}

