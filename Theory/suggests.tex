\subsection{Предложени по модификации}

Оба подхода хорошо себя показывают в проблеме поиска ближайшего соседа. Однако Хочется
предложить модификацию того алгоритма, который был предложен Нижегородскими математиками.

Есть гипотеза, что кол-во длинных рёбер, которые образовываются посредством данного алгоритма
недостаточно для эффективной навигации по графу. Кроме того, нужно следить за тем, чтобы данные
шли случайно, что тоже иногда может быть не очень удобно.

Для решения этой проблемы, предлагаю объединить данный подход с классическим методом построения
тесного графа Уоттса–Строгаца. Идея заключается в следующем:

\begin{itemize}
    \item [1)] Жадным поиском ищем окружение вершины v, которую хотим добавить
    \item [2)] Выбираем k(параметр) ближайших вершин.
    \item [3)] Связываем вершину v лишь с некоторыми из выбранных вершин. Вероятность ребра 
    теперь будет являться отдельным параметром и задаваться заранее
    \item [4)] Каждое непроведённое ребро будет компенсироваться другим. Это новое ребро будет связывать
    нашу вершину v с другой абсолютно случайно выбранной вершиной в графе.
\end{itemize}

В практической части, я собираюсь проверить эффективность данного подхода в сравнении с другими методами.
А также подобрать параметр p, при котром модель будет давать наилучшие результаты.