\section{Заключение}

В заключение хочется сказать, что все необходимы задачи были выполнены:
\begin{itemize}
    \item Были определены основные свойства для графа для оптимального поиска ближайшего соседа
    \item Был полностью изучен феномен тесного мира
    \item Были изучены 2 структуры, которые опираются на этот феномен
    \item Были расписаны их преимущества и недостатки
    \item Были разработаны очень удобные абстракции для работы с графовыми структурами
    \item Был проведён сравнительный анализ структур
\end{itemize}

Обе эти структуры показали свою эффективность в сравнении со случайным графом, за короткое 
кол-во путей они давали результат с очень малой ошибкой, что показывает их эффективность
и применимость. Однако хочется сделать небольшую пометку, построение графа Клайнберга занимает очень
много времени($O(n^2)$) Поэтому его алгоритм лучше использовать, как идею для внедрения в другие типы графов. 
Под идеей имеется в виду расстановка рёбер, опираясь на распределение. Алгоритм построения графа,
который изучали Нижегородские математики, работает достаточно эффективно как по времени, так и по памяти,
поэтому он применим в чистом виде. 

Реализованные структуры очень легко масштабируются, можно внедрить ещё огромное кол-во других абстракций, которые
без "костылей" впишутся в общую структуру классов. Поэтому данный проект можно продолжать развивать,
я планирую сделать свою библиотеку для работы с разными графовыми структурами. Возможно, со
специальными абстракциями для тестирования.